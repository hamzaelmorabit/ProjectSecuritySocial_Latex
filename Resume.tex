\chapter*{Résume}
{\large
De nos jours, les technologies de l’information ont envahi tous les domaines y compris education, transport, logistique, sport, etc. \\
L’objectif de ce projet est de tirer profit des facilités offertes par les technologies de l’information pour l’amélioration des services médicaux. \\
Plus précisément, le but est d’améliorer l’interaction entre les médecins et les patients. Cet amélioration est assuré via la mise en place
d’une application mobile permettant aux médecins de bien gérer et suivre l’état de santé de leurs patients. Au niveau des patients, l’application
leur permettra d’avoir toujours un accès à leurs dossiers médicales. \\

Dans ce présent rapport, nous allons décrire les fonctionnalités de l’application en question, ainsi que le processus de développement suivie. Au niveau technologique, le projet a été implémenté en Java sous forme d’une application Android avec une base de données nonSQL Firebase.
}
\addcontentsline{toc}{chapter}{Résume}

\vspace{3cm}

{\large\textbf{Mots-clés:}}
e-health, Dossier médicale, Application mobile, JAVA, ANDROID, Firebase
